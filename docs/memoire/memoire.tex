% !TEX encoding = UTF-8 Unicode
\documentclass[11p, a4papert]{article}

\usepackage{rotating}
\usepackage[utf8]{inputenc}
\usepackage[T1]{fontenc}
\usepackage[francais]{babel}
\usepackage{multicol}
\usepackage{amsmath}
\usepackage{geometry}
\usepackage{xcolor}
\usepackage[some]{background}
\usepackage{graphics}
\usepackage{graphicx}
\usepackage{stmaryrd}
\DeclareMathOperator{\sinc}{sinc}
\usepackage{lmodern}
\usepackage{verbatim}
\usepackage{lipsum}

%Pour faire des dessins : tikz
\usepackage{pgf}
\usepackage{pgfgantt}
\usepackage{tikz}
\usetikzlibrary{arrows,automata}

\definecolor{titlepagecolor}{cmyk}{1,.60,0,.40}

\backgroundsetup{
scale=1,
angle=0,
opacity=1,
contents={\begin{tikzpicture}[remember picture,overlay]
 \path [fill=titlepagecolor] (current page.west)rectangle (current page.north east); 
 \draw [color=white, very thick] (5,0)--(5,0.5\paperheight);
\end{tikzpicture}}
}

\makeatletter                   
\def\printauthor{%                  
    {\large \@author}}          
\makeatother

\author{%
    Anthony Delannoy \\
    %Department name \\
    \texttt{anthony.delannoy@etu.enseeiht.fr}\vspace{40pt} \\
    Benoit Madiot \\
    %Department name \\
    \texttt{benoit.madiot@etu.enseeiht.fr}\vspace{40pt} \\
    Jérôme Combaniere \\
    %Department name \\
    \texttt{jerome.combaniere@etu.enseeiht.fr} 
    }

%première page
\begin{document}

\begin{titlepage}
\BgThispage
\newgeometry{left=1cm,right=6cm,bottom=3cm}
\vspace*{0.4\textheight}
\noindent
\textcolor{white}{\huge\textbf{\textsf{Méthode de lissage appliquées aux trajectoires ARGOS}}}
\vspace*{3cm}\par
\noindent
\begin{minipage}{0.5\linewidth}
    \begin{flushright}
        \printauthor
    \end{flushright}
\end{minipage} \hspace{15pt}
%
\begin{minipage}{0.02\linewidth}
    \rule{1pt}{175pt}
\end{minipage} \hspace{-10pt}
%
\begin{minipage}{0.63\linewidth}
\vspace{5pt}
    \begin{abstract} 
\lipsum[1]
    \end{abstract}
\end{minipage}
\end{titlepage}
\restoregeometry

%Table des matieres
\newpage
\thispagestyle{empty}
\tableofcontents
\newpage

%%%%%%%Debut du contenu%%%%%%%

\part{Introduction}
\part{Plan de management} % à laisser ?
\part{Code}
\section{RWFormats}
\subsection{Lecture des fichiers}
\subsubsection{Mise en format commun}

\paragraph{Format d'une transmission}
Les éléments essentiels d'une transmission sont stockés dans un dictionnaire respectant l'architecture ci-dessous. Les éléments sont tous des strings sauf "date" qui est un dictionnaire.
\vspace{1cm}
\begin{center}
$uneTransmission = \left\{\begin{array}{c} "date" \\"LC" \\"lat" \\"lon" \\"lat\_image" \\"lon\_image" \\ \end{array}\right\}$
\end{center}

\paragraph{Format du dictionnaire des dates}
\begin{center}
$dicoDate = \left\{\begin{array}{c} "annee" \\"mois" \\"jour" \\"heure" \\"min" \\"sec" \\ \end{array}\right\}$
\end{center}

\paragraph{}
L'utilisation des dictionnaires permettent un rangement et une récupération des données simplifiés, il suffit de connaître la clé de la donnée pour y accéder.

\paragraph{Format des transmissions de tout un fichier}
Comme nous l'avons vu, chaque transmission est stockée comme un dictionnaire. Afin de les ranger dans l'ordre nous les mettons ensuite dans une liste.

\begin{center}
$formatCommun = \left[\begin{array}{cccc} dico[1] & dico[2[ & ... & dico[n] \\ \end{array}\right]$
\end{center}

\subsubsection{Format DIAG}
\subsubsection{Format DS}
\subsubsection{Format CSV}

\subsection{Écriture des fichiers} % à laisser ?
\section{Utilities}
\section{Tracé des cartes}
\part{Conclusion}











\end{document}